% Options for packages loaded elsewhere
\PassOptionsToPackage{unicode}{hyperref}
\PassOptionsToPackage{hyphens}{url}
%
\documentclass[
  ignorenonframetext,
  aspectratio=169]{beamer}
\usepackage{pgfpages}
\setbeamertemplate{caption}[numbered]
\setbeamertemplate{caption label separator}{: }
\setbeamercolor{caption name}{fg=normal text.fg}
\beamertemplatenavigationsymbolsempty
% Prevent slide breaks in the middle of a paragraph
\widowpenalties 1 10000
\raggedbottom
\setbeamertemplate{part page}{
  \centering
  \begin{beamercolorbox}[sep=16pt,center]{part title}
    \usebeamerfont{part title}\insertpart\par
  \end{beamercolorbox}
}
\setbeamertemplate{section page}{
  \centering
  \begin{beamercolorbox}[sep=12pt,center]{part title}
    \usebeamerfont{section title}\insertsection\par
  \end{beamercolorbox}
}
\setbeamertemplate{subsection page}{
  \centering
  \begin{beamercolorbox}[sep=8pt,center]{part title}
    \usebeamerfont{subsection title}\insertsubsection\par
  \end{beamercolorbox}
}
\AtBeginPart{
  \frame{\partpage}
}
\AtBeginSection{
  \ifbibliography
  \else
    \frame{\sectionpage}
  \fi
}
\AtBeginSubsection{
  \frame{\subsectionpage}
}
\usepackage{amsmath,amssymb}
\usepackage{lmodern}
\usepackage{ifxetex,ifluatex}
\ifnum 0\ifxetex 1\fi\ifluatex 1\fi=0 % if pdftex
  \usepackage[T1]{fontenc}
  \usepackage[utf8]{inputenc}
  \usepackage{textcomp} % provide euro and other symbols
\else % if luatex or xetex
  \usepackage{unicode-math}
  \defaultfontfeatures{Scale=MatchLowercase}
  \defaultfontfeatures[\rmfamily]{Ligatures=TeX,Scale=1}
\fi
\usetheme[]{Berlin}
% Use upquote if available, for straight quotes in verbatim environments
\IfFileExists{upquote.sty}{\usepackage{upquote}}{}
\IfFileExists{microtype.sty}{% use microtype if available
  \usepackage[]{microtype}
  \UseMicrotypeSet[protrusion]{basicmath} % disable protrusion for tt fonts
}{}
\makeatletter
\@ifundefined{KOMAClassName}{% if non-KOMA class
  \IfFileExists{parskip.sty}{%
    \usepackage{parskip}
  }{% else
    \setlength{\parindent}{0pt}
    \setlength{\parskip}{6pt plus 2pt minus 1pt}}
}{% if KOMA class
  \KOMAoptions{parskip=half}}
\makeatother
\usepackage{xcolor}
\IfFileExists{xurl.sty}{\usepackage{xurl}}{} % add URL line breaks if available
\IfFileExists{bookmark.sty}{\usepackage{bookmark}}{\usepackage{hyperref}}
\hypersetup{
  pdftitle={Authority After the Tempest},
  pdfauthor={Kevin Morris},
  hidelinks,
  pdfcreator={LaTeX via pandoc}}
\urlstyle{same} % disable monospaced font for URLs
\newif\ifbibliography
\setlength{\emergencystretch}{3em} % prevent overfull lines
\providecommand{\tightlist}{%
  \setlength{\itemsep}{0pt}\setlength{\parskip}{0pt}}
\setcounter{secnumdepth}{-\maxdimen} % remove section numbering
\ifluatex
  \usepackage{selnolig}  % disable illegal ligatures
\fi
\newlength{\cslhangindent}
\setlength{\cslhangindent}{1.5em}
\newlength{\csllabelwidth}
\setlength{\csllabelwidth}{3em}
\newenvironment{CSLReferences}[2] % #1 hanging-ident, #2 entry spacing
 {% don't indent paragraphs
  \setlength{\parindent}{0pt}
  % turn on hanging indent if param 1 is 1
  \ifodd #1 \everypar{\setlength{\hangindent}{\cslhangindent}}\ignorespaces\fi
  % set entry spacing
  \ifnum #2 > 0
  \setlength{\parskip}{#2\baselineskip}
  \fi
 }%
 {}
\usepackage{calc}
\newcommand{\CSLBlock}[1]{#1\hfill\break}
\newcommand{\CSLLeftMargin}[1]{\parbox[t]{\csllabelwidth}{#1}}
\newcommand{\CSLRightInline}[1]{\parbox[t]{\linewidth - \csllabelwidth}{#1}\break}
\newcommand{\CSLIndent}[1]{\hspace{\cslhangindent}#1}

\title{Authority After the Tempest}
\subtitle{Hurricane Michael and the 2018 Elections}
\author{Kevin Morris}
\date{Comparative Politics Workshop, 9/23/2021}
\institute{Brennan Center for Justice / CUNY Graduate Center, Sociology}

\begin{document}
\frame{\titlepage}
\begin{abstract}
\href{mailto:kevin.morris@nyu.edu}{\nolinkurl{kevin.morris@nyu.edu}}
\end{abstract}

\begin{frame}{Hurricanes and other Climate Disasters are Becoming More
Common}
\protect\hypertarget{hurricanes-and-other-climate-disasters-are-becoming-more-common}{}
\end{frame}

\begin{frame}{Hurricanes and other Climate Disasters are Becoming More
Common}
\protect\hypertarget{hurricanes-and-other-climate-disasters-are-becoming-more-common-1}{}
\end{frame}

\begin{frame}{Hurricanes and other Climate Disasters are Becoming More
Common}
\protect\hypertarget{hurricanes-and-other-climate-disasters-are-becoming-more-common-2}{}
\end{frame}

\begin{frame}{Hurricane Michael hits Florida Panhandle on October 27,
2018}
\protect\hypertarget{hurricane-michael-hits-florida-panhandle-on-october-27-2018}{}
\end{frame}

\begin{frame}{Florida Governor Issues Executive Order 18-283}
\protect\hypertarget{florida-governor-issues-executive-order-18-283}{}
\end{frame}

\begin{frame}{Our Question: Can Administrative Changes Prevent Turnout
Losses?}
\protect\hypertarget{our-question-can-administrative-changes-prevent-turnout-losses}{}
\begin{itemize}[<+->]
\tightlist
\item
  The literature on weather indicates that bad weather reduces turnout
  (e.g. Cooperman 2017; Hansford and Gomez 2010; Rallings, Thrasher, and
  Borisyuk 2003)\ldots{} but is this all that informative?
\end{itemize}

\begin{itemize}[<+->]
\tightlist
\item
  Work on other storms is perhaps more illustrative: Stein (2015) finds
  that Superstorm Sandy in 2012 decreased turnout, \emph{except} where
  early in-person voting was available. Kitamura and Matsubayashi (2021)
  finds that voters act dynamically, voting earlier as Typhoon Lan
  approached Japan in 2017.
\end{itemize}

\begin{itemize}[<+->]
\tightlist
\item
  Polling place consolidation, however, might exacerbate this (e.g.
  Brady and McNulty 2011. There are many others).
\end{itemize}

\begin{itemize}[<+->]
\tightlist
\item
  Morris and Miller (2021) shows that polling place consolidation in the
  face of another emergency---COVID-19---had major depressive effects.
\end{itemize}
\end{frame}

\begin{frame}{Data and Methods}
\protect\hypertarget{data-and-methods}{}
\begin{itemize}[<+->]
\tightlist
\item
  We largely leverage a matched-DiD design, where voters in the covered
  counties are matched with voters elsewhere along a battery of relevant
  characteristics, including their past turnout.
\item
  But to separate the administrative effect from the weather effect, we
  need to somehow control for weather.
\item
  That's going to involve a matched-DiD that also incorporates a
  regression discontinuity in space.
\item
  We also look at the vote behavior of folks whose polling places were
  suddenly moved much further away.
\end{itemize}
\end{frame}

\begin{frame}{Average Marginal Effect of the Hurricane}
\protect\hypertarget{average-marginal-effect-of-the-hurricane}{}
\end{frame}

\begin{frame}{Is this influenced by rainfall?}
\protect\hypertarget{is-this-influenced-by-rainfall}{}
\end{frame}

\begin{frame}{How about polling place closures?}
\protect\hypertarget{how-about-polling-place-closures}{}
\end{frame}

\begin{frame}{Decomposing Administrative and Weather Treatments}
\protect\hypertarget{decomposing-administrative-and-weather-treatments}{}
\end{frame}

\begin{frame}{Weather and Administrative Treatments}
\protect\hypertarget{weather-and-administrative-treatments}{}
\end{frame}

\begin{frame}{County-Specific Effects}
\protect\hypertarget{county-specific-effects}{}
\end{frame}

\begin{frame}{County-Specific Effects}
\protect\hypertarget{county-specific-effects-1}{}
\end{frame}

\begin{frame}{How did changed distance to your polling place change
\emph{how} you voted?}
\protect\hypertarget{how-did-changed-distance-to-your-polling-place-change-how-you-voted}{}
\end{frame}

\begin{frame}{Takeaways}
\protect\hypertarget{takeaways}{}
\begin{itemize}[<+->]
\tightlist
\item
  We provide evidence that, even in the face of a Category 5 hurricane,
  big turnout effects can be avoided
\item
  This means that election administrators can't hide behind the
  weather---how they responded to the storm in 2018 had big
  participatory consequences
\item
  The distributed nature of American election systems means that the
  effects of weather emergencies are going to be hyper-local.
\item
  My question for YOU ALL: how can this translate into an international
  context? Can this inform how other countries prepare for climate
  disasters? What is the importance of the US's highly federalized
  election administration regime?
\end{itemize}
\end{frame}

\begin{frame}{Thanks!}
\protect\hypertarget{thanks}{}
\href{mailto:kevin.morris@nyu.edu}{\nolinkurl{kevin.morris@nyu.edu}}
\end{frame}

\begin{frame}[allowframebreaks]{References}
\protect\hypertarget{references}{}
\hypertarget{refs}{}
\begin{CSLReferences}{1}{0}
\leavevmode\hypertarget{ref-Brady2011}{}%
Brady, Henry, and John McNulty. 2011. {``Turning Out to Vote: {The
Costs} of {Finding} and {Getting} to the {Polling Place}.''}
\emph{American Political Science Review} 105 (1): 115--34.

\leavevmode\hypertarget{ref-Cooperman2017}{}%
Cooperman, Alicia. 2017. {``Randomization {Inference} with {Rainfall
Data}: {Using Historical Weather Patterns} for {Variance Estimation}.''}
\emph{Political Analysis} 25 (3): 277--88.

\leavevmode\hypertarget{ref-Hansford2010}{}%
Hansford, Thomas, and Brad Gomez. 2010. {``Estimating the {Electoral
Effects} of {Voter Turnout}.''} \emph{American Political Science Review}
104: 268--88.

\leavevmode\hypertarget{ref-Kitamura2021}{}%
Kitamura, Shuhei, and Tetsuya Matsubayashi. 2021. {``Dynamic
{Voting}.''} \emph{Working Paper}, January.
\url{https://doi.org/10.2139/ssrn.3630064}.

\leavevmode\hypertarget{ref-Morris2021}{}%
Morris, Kevin, and Peter Miller. 2021. {``Voting in a {Pandemic}:
{COVID}-19 and {Primary Turnout} in {Milwaukee}, {Wisconsin}.''}
\emph{Urban Affairs Review}, April, 10780874211005016.
\url{https://doi.org/10.1177/10780874211005016}.

\leavevmode\hypertarget{ref-Rallings2003}{}%
Rallings, Colin, Michael Thrasher, and Roman Borisyuk. 2003. {``Seasonal
{Factors}, Voter Fatigue, and the Costs of Voting.''} \emph{Electoral
Studies} 22: 65--79.

\leavevmode\hypertarget{ref-Stein2015}{}%
Stein, Robert. 2015. {``Election {Administration During National
Disasters} and {Emergencies}: {Hurricane Sandy} and the 2012
{Election}.''} \emph{Election Law Journal} 14: 66--73.

\end{CSLReferences}
\end{frame}

\end{document}
