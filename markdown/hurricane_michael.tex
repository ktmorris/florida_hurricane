% Options for packages loaded elsewhere
\PassOptionsToPackage{unicode}{hyperref}
\PassOptionsToPackage{hyphens}{url}
%
\documentclass[
  12pt,
]{article}
\usepackage{lmodern}
\usepackage{amsmath}
\usepackage{ifxetex,ifluatex}
\ifnum 0\ifxetex 1\fi\ifluatex 1\fi=0 % if pdftex
  \usepackage[T1]{fontenc}
  \usepackage[utf8]{inputenc}
  \usepackage{textcomp} % provide euro and other symbols
  \usepackage{amssymb}
\else % if luatex or xetex
  \usepackage{unicode-math}
  \defaultfontfeatures{Scale=MatchLowercase}
  \defaultfontfeatures[\rmfamily]{Ligatures=TeX,Scale=1}
\fi
% Use upquote if available, for straight quotes in verbatim environments
\IfFileExists{upquote.sty}{\usepackage{upquote}}{}
\IfFileExists{microtype.sty}{% use microtype if available
  \usepackage[]{microtype}
  \UseMicrotypeSet[protrusion]{basicmath} % disable protrusion for tt fonts
}{}
\makeatletter
\@ifundefined{KOMAClassName}{% if non-KOMA class
  \IfFileExists{parskip.sty}{%
    \usepackage{parskip}
  }{% else
    \setlength{\parindent}{0pt}
    \setlength{\parskip}{6pt plus 2pt minus 1pt}}
}{% if KOMA class
  \KOMAoptions{parskip=half}}
\makeatother
\usepackage{xcolor}
\IfFileExists{xurl.sty}{\usepackage{xurl}}{} % add URL line breaks if available
\IfFileExists{bookmark.sty}{\usepackage{bookmark}}{\usepackage{hyperref}}
\hypersetup{
  pdftitle={Authority After the Tempest: Hurricane Michael and the 2018 Elections},
  pdfauthor={Kevin Morris; Peter Miller},
  hidelinks,
  pdfcreator={LaTeX via pandoc}}
\urlstyle{same} % disable monospaced font for URLs
\usepackage[margin=1in]{geometry}
\usepackage{longtable,booktabs}
\usepackage{calc} % for calculating minipage widths
% Correct order of tables after \paragraph or \subparagraph
\usepackage{etoolbox}
\makeatletter
\patchcmd\longtable{\par}{\if@noskipsec\mbox{}\fi\par}{}{}
\makeatother
% Allow footnotes in longtable head/foot
\IfFileExists{footnotehyper.sty}{\usepackage{footnotehyper}}{\usepackage{footnote}}
\makesavenoteenv{longtable}
\usepackage{graphicx}
\makeatletter
\def\maxwidth{\ifdim\Gin@nat@width>\linewidth\linewidth\else\Gin@nat@width\fi}
\def\maxheight{\ifdim\Gin@nat@height>\textheight\textheight\else\Gin@nat@height\fi}
\makeatother
% Scale images if necessary, so that they will not overflow the page
% margins by default, and it is still possible to overwrite the defaults
% using explicit options in \includegraphics[width, height, ...]{}
\setkeys{Gin}{width=\maxwidth,height=\maxheight,keepaspectratio}
% Set default figure placement to htbp
\makeatletter
\def\fps@figure{htbp}
\makeatother
\usepackage[normalem]{ulem}
% Avoid problems with \sout in headers with hyperref
\pdfstringdefDisableCommands{\renewcommand{\sout}{}}
\setlength{\emergencystretch}{3em} % prevent overfull lines
\providecommand{\tightlist}{%
  \setlength{\itemsep}{0pt}\setlength{\parskip}{0pt}}
\setcounter{secnumdepth}{5}
\usepackage{rotating}
\usepackage{setspace}
\newcommand{\beginsupplement}{\setcounter{table}{0}  \renewcommand{\thetable}{A\arabic{table}} \setcounter{figure}{0} \renewcommand{\thefigure}{A\arabic{figure}}}
\usepackage{lineno}
\linenumbers
\usepackage{booktabs}
\usepackage{longtable}
\usepackage{array}
\usepackage{multirow}
\usepackage{wrapfig}
\usepackage{float}
\usepackage{colortbl}
\usepackage{pdflscape}
\usepackage{tabu}
\usepackage{threeparttable}
\usepackage{threeparttablex}
\usepackage[normalem]{ulem}
\usepackage{makecell}
\usepackage{xcolor}
\ifluatex
  \usepackage{selnolig}  % disable illegal ligatures
\fi
\newlength{\cslhangindent}
\setlength{\cslhangindent}{1.5em}
\newlength{\csllabelwidth}
\setlength{\csllabelwidth}{3em}
\newenvironment{CSLReferences}[2] % #1 hanging-ident, #2 entry spacing
 {% don't indent paragraphs
  \setlength{\parindent}{0pt}
  % turn on hanging indent if param 1 is 1
  \ifodd #1 \everypar{\setlength{\hangindent}{\cslhangindent}}\ignorespaces\fi
  % set entry spacing
  \ifnum #2 > 0
  \setlength{\parskip}{#2\baselineskip}
  \fi
 }%
 {}
\usepackage{calc}
\newcommand{\CSLBlock}[1]{#1\hfill\break}
\newcommand{\CSLLeftMargin}[1]{\parbox[t]{\csllabelwidth}{#1}}
\newcommand{\CSLRightInline}[1]{\parbox[t]{\linewidth - \csllabelwidth}{#1}\break}
\newcommand{\CSLIndent}[1]{\hspace{\cslhangindent}#1}

\title{Authority After the Tempest: Hurricane Michael and the 2018 Elections}
\author{Kevin Morris\footnote{Researcher, Brennan Center for Justice at NYU School of Law, 120 Broadway Ste 1750, New York, NY 10271 (\href{mailto:kevin.morris@nyu.edu}{\nolinkurl{kevin.morris@nyu.edu}})} \and Peter Miller\footnote{Researcher, Brennan Center for Justice at NYU School of Law, 120 Broadway Ste 1750, New York, NY 10271 (\href{mailto:peter.miller@nyu.edu}{\nolinkurl{peter.miller@nyu.edu}})}}
\date{May 26, 2021}

\begin{document}
\maketitle
\begin{abstract}
Hurricane Michael made landfall in the Florida panhandle 27 days before the 2018 elections. In the aftermath, the governor of Florida issued Executive Order 18-283 granting election officials in 8 impacted counties the autonomy to loosen a variety of voting laws related to early in-person voting, voting by mail ballots, and the number and location of polling places to ensure the orderly conduct of the election. To test the efficacy of the order we deploy a novel research design to separate the effects of the hurricane on turnout from the administrative effects of actions taken by election officials. By leveraging cross-jurisdiction variation in a double-matched, triple-differences model, we show that the Executive Order was not successful at eliminating declining turnout. As administrators loosen mail-voting restrictions in advance of this fall, they must couple these eased restrictions with strong public education campaigns about how voters can take advantage of them. \textbf{Do we need to revise the abstract in light of new results?}
\end{abstract}

\pagenumbering{gobble}
\pagebreak

\pagenumbering{arabic}
\doublespacing

\hypertarget{introduction}{%
\section*{Introduction}\label{introduction}}
\addcontentsline{toc}{section}{Introduction}

As the 2018 elections approached, an unanticipated---but not unprecedented---shape appeared on the Florida horizon: the Category 5 Hurricane Michael.\footnote{The category of the hurricane refers to the maximum sustained wind speed, according to the Saffir-Simpson hurricane wind scale. A Category 5 hurricane sustains winds greater than 157 miles per hour, as measured as the peak 1-minute wind at a height of 33 feet. See \url{https://www.nhc.noaa.gov/pdf/sshws.pdf}.} The hurricane made landfall on October 10, 27 days before the election, and would ultimately cause 16 deaths and 25 billion dollars in damage.\footnote{See \url{https://www.nhc.noaa.gov/data/tcr/AL142018_Michael.pdf}.} Would-be voters in the election were now faced with myriad disruptions to their daily lives; the direct effects of the weather, therefore, likely reduced turnout substantially as the recovery from the hurricane progressed. As professor emeritus Robert Montjoy told NPR in the aftermath of the storm, ``Whether casting a ballot becomes a higher priority than cleaning out the basement, visiting someone in the hospital, or all the other demands\ldots You certainly expect a lower turnout for those reasons'' (\protect\hyperlink{ref-Parks2018}{Parks 2018}).

The storm also affected the administration of the election itself, as polling places were destroyed and potential mail voters found themselves temporarily residing at addresses other than those at which they were registered. The governor of Florida issued Executive Order 18-283\footnote{See \url{https://www.flgov.com/wp-content/uploads/2018/10/SLT-BIZHUB18101809500.pdf}.} as a means to counteract the widespread effects of the hurricane on October 18. Executive Order 18-283 sought to offset the administrative barriers to voting by allowing election administrators in 8 counties in Florida affected by the hurricane to flexibly respond to the damage wrought by the storm. Specifically, Executive Order 18-283 allowed administrators to add early voting locations; begin early voting 15 days before the general election (4 days after the Executive Order was issued), and continue until the day of the election; to accept vote-by-mail requests to addresses other than a voter's registered address; to send vote-by-mail ballots by forwardable mail; to deliver vote-by-mail ballots to electors or electors' immediate family members on election day without an affidavit; to relocate or consolidate polling places; and required poll watchers to be registered by the second Friday before the general election. The Executive Order covered Bay, Calhoun, Franklin, Gadsden, Gulf, Jackson, Liberty, and Washington Counties.

This paper sets out to answer a number of questions: what was the total depressive effect of the hurricane on turnout in the election? Did Executive Order 18-283 effectively offset these effects? More specifically, did easing mail-balloting and early voting rules reduce the impact of closed or moved polling places? We propose a novel research design to investigate these interrelated questions---what we are calling a double-matched, triple-difference model---and then demonstrate that the hurricane significantly reduced turnout and that responses to the hurricane by local election officials were unable to overcome the devastation of the hurricane. We conclude with some thoughts about how the instance of Hurricane Michael can inform the conduct of elections under other natural disasters likely to occur in the future.

\hypertarget{literature-review}{%
\section*{Literature Review}\label{literature-review}}
\addcontentsline{toc}{section}{Literature Review}

The institutional and weather conditions of Hurricane Michael make it ripe for studying the interactive effects of severe weather, polling place siting, and administrative regimes. Understanding these relationships are will be of key importance in the coming years as climate change leads to increasingly strong storms. This is doubly true in the American context, where election days are held at the end of hurricane season. Although little work has explored these effects, we here consider how Florida's permissive early voting regime, the Executive Order's allowance of polling place consolidation, and severe weather might have collectively structured turnout in 2018. Our general conclusion is that, early voting could have likely served as a ``relief valve'' on the pressures introduced by the inclement weather, but that polling place consolidation likely had major, negative turnout effects.

\hypertarget{early-voting-and-inclement-weather}{%
\subsection*{Early Voting and Inclement Weather}\label{early-voting-and-inclement-weather}}
\addcontentsline{toc}{subsection}{Early Voting and Inclement Weather}

It is well established that inclement weather on election day reduces turnout in both the American (\protect\hyperlink{ref-Cooperman2017}{Cooperman 2017}; \protect\hyperlink{ref-Hansford2010}{Hansford and Gomez 2010, 269}) and international context (\protect\hyperlink{ref-Rallings2003}{Rallings, Thrasher, and Borisyuk 2003}), especially in noncompetitive and general elections (\protect\hyperlink{ref-Gatrell2002}{Gatrell and Bierly 2002}; \protect\hyperlink{ref-Fraga2010}{Fraga and Hersh 2010}). A recent study based on Irish parliamentary elections indicates that this is especially true in densely populated areas (\protect\hyperlink{ref-Garcia-Rodriguez2020}{Garcia-Rodriguez and Redmond 2020}). This is perhaps unsurprising: severe weather reduces turnout increases the opportunity cost of voting. Driving to a polling place or, worse, waiting outside in line to vote is obviously much more costly in severe weather events. As the above quote from professor emeritus Robert Montjoy makes evident, a natural disaster can increase burdens on households even if it strikes before election day, perhaps leaving them less likely to learn about the candidates, locate their polling place, and cast a ballot.

Although Floridians in the panhandle faced a Category 5 hurricane in 2018, the hurricane arrived against the backdrop of Florida's permissive early voting infrastructure. In recent years, as many as XX\% of Floridians have cast their ballots early in-person, prior to election day. It seems plausible that this availability could have sufficiently reduced the cost of voting to offset some of the negative effects associated with the storm. While research on the impact of early in-person voting on turnout in non-emergency times has returned mixed results (see, for instance, \protect\hyperlink{ref-Ricardson1996}{Ricardson and Neeley 1996}; \protect\hyperlink{ref-Larocca2011}{Larocca and Klemanski 2011}; \protect\hyperlink{ref-Burden2014}{Burden et al. 2014}; \protect\hyperlink{ref-Kaplan2020}{Kaplan and Yuan 2020}), a growing body of literature suggests that the availability of early in-person voting might be important in the context of severe weather. One study in Sweden, for instance, found no significant turnout effects of rain on election day, which they attribute to Sweden's permissive early voting regime (\protect\hyperlink{ref-Persson2014}{Persson, Sundell, and Öhrvall 2014, 337}); voters were able to avoid an incoming storm by casting a ballot in advance.

Furthermore, and most relevant to our study of Hurricane Michael, is the effect of Typhoon Lan\footnote{Lan was the equivalent of a Category 4 hurricane, featuring wind speeds of between 130 and 156 miles per hour.} on turnout in the 2017 House of Representatives election in Japan. The typhoon made landfall the day after election day, though it appears voters behaved dynamically as the typhoon approached: voters were more likely to vote early, or earlier on the day of the election, as rainfall increased in prefectures in the path of the typhoon (\protect\hyperlink{ref-Kitamura2020}{Kitamura and Matsubayashi 2020}). Of course, we cannot know which individuals who voted early would have voted at the height of the typhoon, and which would have opted to stay home. Nevertheless, it is not unreasonable to assume that the availability of early voting led some voters to participate who would not have been inclined to do so in worse weather.

The experience of Superstorm Sandy in the Northeastern United States in 2012 provides more evidence of the importance of early voting in the face of severe weather. Stein (\protect\hyperlink{ref-Stein2015}{2015, 69}) argues that turnout in counties impacted by Superstorm Sandy decreased by 2.8\% between 2008 and 2012---a full 2\% more than the rest of the country. He finds, however, that counties that provided for early in-person voting actually saw \emph{higher} turnout in 2012 than other comparable counties. It seems that, whatever questions remain about the impact of early in-person voting on turnout in normal times, that such an option may provide a way to recoup some of the lost turnout caused by a natural disaster.

\hypertarget{polling-place-consolidation}{%
\subsection*{Polling Place Consolidation}\label{polling-place-consolidation}}
\addcontentsline{toc}{subsection}{Polling Place Consolidation}

Even as Floridians had access to widespread early in-person voting in 2018, Hurricane Michael and Executive Order 2018-283 allowed for and effected major polling place consolidation in the covered counties. In fact, just 62 of the planned 127 polling places were open across the 8 counties covered by the executive order. Understanding the impact of these consolidations in light of the hurricane is important for understanding the anticipated effect of the storm on turnout---and, in particular, the effect of the executive order.

Although \protect\hyperlink{ref-Stein2015}{Stein} (\protect\hyperlink{ref-Stein2015}{2015}) argues that counties impacted by Superstorm Sandy that consolidated polling places saw higher turnout than those that were effected but did not consolidate their polling places, this result is something of an outlier. In fact, the extant literature is largely consistent in its conclusion that polling place consolidation reduces turnout. Relocating or reducing the number of polling places in turn reduces turnout by imposing new search and transportation costs on voters (\protect\hyperlink{ref-Brady2011}{Brady and McNulty 2011}). A moved polling place reduces turnout in a variety of electoral contexts (\protect\hyperlink{ref-Cantoni2020}{Cantoni 2020}), including local elections (\protect\hyperlink{ref-McNulty2009}{McNulty, Dowling, and Ariotti 2009}; \protect\hyperlink{ref-Haspel2005}{Haspel and Knotts 2005}) as well as national contests (\protect\hyperlink{ref-Kropf2012}{Kropf and Kimball 2012}). Absentee voting is more likely as the distance to the polls increases, but this effect is not large enough to offset the decrease from consolidation itself (\protect\hyperlink{ref-Brady2011}{Brady and McNulty 2011}; \protect\hyperlink{ref-Dyck2005}{Dyck and Gimpel 2005}).

Although there has been little work on the effect of polling place consolidation on turnout in the face of a storm, recent work indicates that last-minute polling place consolidation reduced turnout in another sort of natural disaster---namely, the COVID-19 pandemic in 2020. During the April 2020 primary election in Milwaukee, Wisconsin, the municipality when from nearly 200 to just 5 polling places. \protect\hyperlink{ref-Morris2021}{Morris and Miller} (\protect\hyperlink{ref-Morris2021}{2021}) shows that this consolidation had major, negative turnout effects, even though Wisconsin has a robust absentee voting regime. They conclude: ``Even as many voters transition to vote-by-mail in the face of a pandemic, polling place consolidation can still have disenfranchising effects'' (\protect\hyperlink{ref-Morris2021}{Morris and Miller 2021, 13}).

Grounding our analyses of the effects of Hurricane Michael gives us some expectations as to how the hurricane will alter voting behavior. We expect the direct, weather-related effects of the hurricane to reduce turnout. The administrative effects will push in opposite directions. On the one hand, consolidated polling places likely imposed costs on voters, reducing turnout above-and-beyond the direct effects of weather. On the other hand, the relief valve offered by early voting may recover \emph{some but not all} of these displaced voters. This is, of course, not to claim that the local officials in the path of the hurricane sought to reduce turnout. Rather, the work of administering an election --- even under the best of circumstances --- is difficult \textbf{CITATION?}. The extraordinary impact of a Category 5 hurricane is perhaps simply too much for election administrators to incorporate into their efforts to conduct a secure and inclusive election.

\hypertarget{research-design-and-expectations}{%
\section*{Research Design and Expectations}\label{research-design-and-expectations}}
\addcontentsline{toc}{section}{Research Design and Expectations}

We expect that Hurricane Michael depressed turnout in the 2018 midterm election via two causal mechanisms: weather effects, and administrative effects. Throughout our analyses, we examine the effects of the hurricane on voters registered as of the 2018 election. Put differently, we do not test the turnout of \emph{eligible citizens}. Conditioning turnout on registration status raises important questions when the treatment might influence registration (see \protect\hyperlink{ref-Nyhan2017}{Nyhan, Skovron, and Titiunik 2017}). That is likely the case here: as we demonstrate in the Supplemental Information, it seems probable that Hurricane Michael reduced registrations in the days before the registration deadline. Our models cannot capture these turnout effects; as such, our estimated negative treatment effects should be considered conservative, as we are not measuring the turnout of individuals whose registration---and subsequent participation---was impeded by the storm.

\hypertarget{estimating-the-overall-effects-of-the-hurricane}{%
\subsection*{Estimating the Overall Effects of the Hurricane}\label{estimating-the-overall-effects-of-the-hurricane}}
\addcontentsline{toc}{subsection}{Estimating the Overall Effects of the Hurricane}

We begin by testing the average marginal effect (AME) of Hurricane Michael on turnout. The AME is the net effect of both the weather and the administrative effects on individual-level turnout. Our central identification strategy involves the use of difference-in-differences models. We use voter-file data from L2 Political to estimate individual-level turnout and to control for individual-level characteristics and the latitude and longitude of each voter's residential address. L2 uses models to predict individual race / ethnicity and voters' sex but these characteristics are available in self-reported form in the raw voter-file available from the state; as such, we pull sex and race / ethnicity from the publicly available voter file. The L2 data is based on the February 8, 2019, version of the raw voter file, the same file from which we pull race / ethnicity and sex.

In addition to the individual-level characteristics from the voter file, we also proxy each voter's exposure to Hurricane Michael using rainfall data. The National Oceanic and Atmospheric Administration (NOAA) makes daily rainfall data available for some 13,000 stations around the United States. At each weather station, we use the \texttt{rnoaa} (\protect\hyperlink{ref-Chamberlain2021}{Chamberlain 2021}) package to measure the amount of rain that fell between October 10 and November 6 in 2018 relative to the average rainfall in that period from 2000 to 2017. Voters' individual exposure to rainfall is calculated as the average of the three closest weather stations, inversely weighted by distance.

Finally, we incorporate information garnered from public records requests sent to each of the 8 treated counties about the number of polling places they closed due to Hurricane Michael. Three counties (Calhoun, Gadsden, and Liberty) closed no polling places, while a fourth (Franklin) actually added added an additional polling place. The other four covered counties cut the number of polling places by at least two-thirds. We leverage this heterogeneity to explore the effect of closed polling places on turnout, and expect the turnout effect of the storm was lower (that is, less negative) in the counties where more polling places were open.

By comparing historical and 2018 turnout for voters in the counties hit by the storm to historical and 2018 turnout of voters elsewhere in the state, we can estimate the AME of the storm on turnout. To ensure a high-quality difference-in-differences specification, we do not include all untreated voters in our control group; rather, we genetically match (\protect\hyperlink{ref-Sekhon2011}{Sekhon 2011}) each treated voter with five untreated voters along a battery of individual- and neighborhood-level characteristics, including past turnout and vote mode. Untreated voters who do not serve as matches are excluded from our models. Although it may seem counterintuitive to exclude data from our models, this matching procedure substantially improves the parallel trends assumptions necessary for a rigorous difference-in-differences analysis (\protect\hyperlink{ref-Sekhon2009}{Sekhon 2009, 496}). As we show in the Supplemental Information, our results are robust whether we do not match, we employ a different matching approach, or utilize entropy balancing.

This design allows us to test our first hypothesis:

\textbf{Hypothesis 1:} Turnout among voters in the eight treated counties was depressed in the 2018 election relative to voters in untreated counties. We expect that the negative AME will be larger in counties that closed more polling places in response to the Executive Order, and where the relative rainfall was higher.

\hypertarget{decomposing-weather-and-administrative-effects}{%
\subsection*{Decomposing Weather and Administrative Effects}\label{decomposing-weather-and-administrative-effects}}
\addcontentsline{toc}{subsection}{Decomposing Weather and Administrative Effects}

To estimate the administrative effect on turnout, we must control for the weather effects encountered by each voter. To do so, we leverage the somewhat arbitrary borders of counties in the Florida Panhandle, an approach similar to that adopted in a different context by \protect\hyperlink{ref-Walker2019}{Walker, Herron, and Smith} (\protect\hyperlink{ref-Walker2019}{2019}). This is often referred to as a geographical regression discontinuity (\protect\hyperlink{ref-Keele2015}{L. J. Keele and Titiunik 2015/ed}). There is no reason to believe that the effects of a hurricane would change dramatically along county borders. We assume, therefore, that voters who lived nearby one another, but on either side of a county border, faced the same weather issues during the 2018 election. Put differently, these voters were identically ``treated'' by the weather effects of the hurricane. Within a narrow buffer around the county border, we can conceive of a voter's county as effectively randomly assigned. Any observed turnout differential, therefore, is attributable \emph{not} to the weather, but the administrative effects of the county in which they happen to live. While all these voters were ``treated'' by the hurricane, only those in the covered counties also received an administrative treatment.

Of course, self-selection around a geographic boundary is entirely possible; as such, conceiving of the administrative boundary as a quasi-random assignment is perhaps too strong of an assumption. Treated and control voters, despite living very near to one another, might differ in meaningful ways. To address this potential problem, we adopt the technique developed by \protect\hyperlink{ref-Keele2015a}{L. Keele, Titiunik, and Zubizarreta} (\protect\hyperlink{ref-Keele2015a}{2015}) by also matching voters on either side of the boundary according to their historical turnout and vote mode. To strengthen the plausibility that these two sets of voters were identically treated by the weather, we also match on each voter's relative rainfall.

By comparing the 2018 turnout of these voters, we can identify the administrative effect of the Executive Order on turnout for the treated voters living within the buffer around the border. By further comparing the turnout of these voters to (matched) voters elsewhere in the state, we can also estimate the weather effects of the storm. We call this a double-matched triple-differences (or difference-in-difference-in-differences) specification. We lay out the specific steps below.

We begin by constructing our set of voters who received an administrative treatment. These voters include all registered voters who live in a county covered by the executive order and within 2.5 miles of a bordering, uncovered county (See Figure \ref{fig:map}). Each treated voter is then matched to one voter who lives in an uncovered county, but within 2.5 miles of a covered county. All of these voters were treated by the weather, but only those in the covered counties were also treated by the administrative changes. Although Calhoun, Franklin, and Gulf Counties were covered by the Executive Order, no voters in these counties live within 2.5 miles of an uncovered county; as such, these voters are not included in these models.

\begin{figure}[h]

{\centering \includegraphics{hurricane_michael_files/figure-latex/map-chunk-1} 

}

\caption{\label{fig:map}Treated and Control Counties with 2.5 Mile Buffer}\label{fig:map-chunk}
\end{figure}

Each of these voters is subsequently matched to five voters elsewhere in the state---that is to say, voters who received neither a weather treatment \emph{nor} an administrative one. This exercise is the second match, and the matches are our control voters.

Table \ref{tab:groups} summarizes the treatment status of our three groups of voters.

\begin{singlespace}
\begin{table}[H]

\caption{\label{tab:groups-treat}\label{tab:groups} Treatment Status for Selected Voters}
\centering
\begin{tabular}[t]{>{\raggedright\arraybackslash}p{15em}ll}
\toprule
\multicolumn{1}{c}{ } & \multicolumn{2}{c}{Treatment Received} \\
\cmidrule(l{3pt}r{3pt}){2-3}
Group & Administrative & Weather\\
\midrule
\cellcolor{gray!6}{Selected Voters in Covered Counties} & \cellcolor{gray!6}{Yes} & \cellcolor{gray!6}{Yes}\\
Selected Voters in Uncovered Counties in Panhandle & No & Yes\\
\cellcolor{gray!6}{Selected Voters Elsewhere} & \cellcolor{gray!6}{No} & \cellcolor{gray!6}{No}\\
\bottomrule
\end{tabular}
\end{table}
\end{singlespace}

Having constructed our pool of voters, we run a triple-differences model. This triple-differences model is, in effect, two simultaneous difference-in-differences models. The model estimates whether 2018 was associated with depressed turnout for voters treated only by the weather vis-à-vis the controls who received no treatment. Because these treated voters lived in counties not covered by the Executive Order, we assume that they faced no administrative effects from the storm. Any observed difference between these groups is therefore the weather effect for all voters treated by the weather, regardless of whether they received an additional, administrative treatment.

The model also estimates turnout differences between voters treated by the weather and administrative effects, and those treated only by the weather. Because we assume these closely-located voters faced identical weather effects, any difference them is the administrative effect on turnout of living in a covered county.

The double-matched triple-differences model allows us to test our second and third hypotheses:

\textbf{Hypothesis 2:} We expect that the hurricane had negative weather effects for voters who lived just outside of covered counties.

\textbf{Hypothesis 3:} We expect that the administrative effect will be largely driven by the number of polling places each county consolidated, other things equal. Where many polling places were closed we anticipate a large, negative administrative effect (\protect\hyperlink{ref-Morris2021}{Morris and Miller 2021}). In contrast, where most polling places remained open, we expect small negative or small positive administrative effects.

In short, our empirical strategy incorporates matching, difference-in-differences, and a regression discontinuity, three powerful tools for establishing causality.

\hypertarget{vote-mode}{%
\subsection*{Vote Mode}\label{vote-mode}}
\addcontentsline{toc}{subsection}{Vote Mode}

After estimating the double-matched triple-differences model, we turn to vote-mode within the treated counties. \sout{We submitted public records requests to each of the eight counties covered by the Executive Order requesting the planned and actual location of each polling place. The changes in polling places are summarized in Table \ref{tab:moved-pps}.} To test whether the Executive Order shifted vote mode from in-person to mail voting in the treated counties, we begin by calculating how far each voter lived from the closest planned polling place, and how far she lived from the closest polling place that was actually open on election day. Using the registered voter file, we can tell not only \emph{whether} a voter participated, but also \emph{how} they participated. Using a multinomial logistic regression, we test whether the difference between the planned and actual distance-to-polling-place were associated with vote-mode in 2018. This specification allows us to test our final hypothesis:

\textbf{Hypothesis 4}: As the difference between the actual and planned distance to the closest polling place increased for voters, they were more likely to vote absentee and to abstain from voting, all else being held equal.

\hypertarget{results}{%
\section*{Results}\label{results}}
\addcontentsline{toc}{section}{Results}

\hypertarget{overall-turnout-effects}{%
\subsection*{Overall Turnout Effects}\label{overall-turnout-effects}}
\addcontentsline{toc}{subsection}{Overall Turnout Effects}

We begin by matching each registered voter in the eight treated counties to five untreated voters elsewhere in the state using a nearest neighbor approach. We use a genetic algorithm to determine the weight each characteristic should receive for the matching procedure (\protect\hyperlink{ref-Sekhon2011}{Sekhon 2011}).\footnote{Due to computing constraints, the matching weights were constructed using a one percent random sample stratified by treatment status. The weights derived from the genetic algorithm are then used to perform the nearest-neighbor match for all treated voters.} The individual-level characteristics come directly from the L2 and the registered voter file. The two neighborhood-level characteristics included --- median income and share of the population with some collegiate education --- are estimated at the block group level, and come from the ACS 5-year estimates ending with 2018. Ties are not broken, which means that some treated voters are assigned more than five control voters; the weights used in the regressions below are adjusted accordingly.

Although the treated counties were at the center of the storm, nearby counties might have also been negatively impacted by the storm. Therefore, voters who live in the counties that border the treated counties are excluded as potential controls. These include Walton, Holmes, Wakulla, and Leon Counties. According to public records requests we filed, none of these counties reduced polling places or early voting days because of the hurricane.

Table \ref{tab:full-bal} demonstrates the results of this matching procedure. As Table \ref{tab:full-bal} makes clear, voters in the affected counties were considerably more likely to be white and identify as Republicans, and live in lower-income neighborhoods, than voters in the rest of the state. The post-match control group, however, looks substantially similar to the treated voters.

\begin{singlespace}
\begin{table}[!h]

\caption{\label{tab:balance-tab-full}\label{tab:full-bal} Balance Table for Statewide Matching}
\centering
\resizebox{\linewidth}{!}{
\begin{tabular}[t]{lllllrrrr}
\toprule
\multicolumn{1}{c}{ } & \multicolumn{2}{c}{Means: Unmatched Data} & \multicolumn{2}{c}{Means: Matched Data} & \multicolumn{4}{c}{Percent Improvement} \\
\cmidrule(l{3pt}r{3pt}){2-3} \cmidrule(l{3pt}r{3pt}){4-5} \cmidrule(l{3pt}r{3pt}){6-9}
 & Treated & Control & Treated & Control & Mean Diff & eQQ Med & eQQ Mean & eQQ Max\\
\midrule
\%White & 76.5\% & 62.3\% & 76.5\% & 76.5\% & 100.00 & 100.00 & 100.00 & 100.00\\
\% Black & 17.1\% & 13.1\% & 17.1\% & 17.1\% & 100.00 & 100.00 & 100.00 & 100.00\\
\% Latino & 2.1\% & 17.4\% & 2.1\% & 2.1\% & 100.00 & 100.00 & 100.00 & 100.00\\
\% Asian & 1.0\% & 2.0\% & 1.0\% & 1.0\% & 100.00 & 100.00 & 100.00 & 100.00\\
\% Female & 52.5\% & 52.4\% & 52.5\% & 52.5\% & 100.00 & 100.00 & 100.00 & 100.00\\
\% Male & 45.8\% & 44.9\% & 45.8\% & 45.8\% & 100.00 & 100.00 & 100.00 & 100.00\\
Age & 52.2 & 52.5 & 52.2 & 52.2 & 98.54 & 96.68 & 97.36 & 96.17\\
\% Democrat & 39.2\% & 37.1\% & 39.2\% & 39.2\% & 100.00 & 100.00 & 100.00 & 100.00\\
\% Republican & 43.6\% & 35.0\% & 43.6\% & 43.6\% & 100.00 & 100.00 & 100.00 & 100.00\\
\% with Some College & 69.0\% & 75.1\% & 69.0\% & 69.0\% & 99.77 & 99.00 & 98.05 & 88.66\\
Median Income & \$50,643 & \$62,941 & \$50,643 & \$50,654 & 99.91 & 98.11 & 96.89 & 86.56\\
\bottomrule
\end{tabular}}
\end{table}
\end{singlespace}

Figure \ref{fig:full-to} plots the turnout in the past few elections for our treated and control voters. The left-hand panel shows the turnout of all voters. In the right-hand panel, we plot the turnout of treated voters and only their controls. As Figure \ref{fig:full-to} makes clear, turnout in the treated counties was consistently higher than the rest of the state---until 2018, when the hurricane hit. In the right-hand panel, we see that the matching procedure was successful at reducing historical differences between treated and control voters, and that there was a substantial, negative treatment effect in 2018.

\begin{figure}[h]

{\centering \includegraphics{hurricane_michael_files/figure-latex/full-to-chunk-1} 

}

\caption{\label{fig:full-to}General Election Turnout for Treated and Control Voters, 2010 -- 2018}\label{fig:full-to-chunk}
\end{figure}

Table \ref{tab:full-dind} formalizes the right-hand panel of Figure \ref{fig:full-to} into a differences-in-differences regression. We employ an ordinary least squares specification. The dependent variable takes the value 1 if a voter cast a ballot in a given year, and 0 if she did not. In each model, \emph{Treated × 2018} estimates the causal (net) effect of Hurricane Michael on turnout for treated voters. Model 2 also includes the characteristics on which the voters were matched. Model 3 adds a measure for congressional district competitiveness. Because this variable is ``downstream'' of treatment --- that is to say, the effect of the hurricane could have impacted the competitiveness of certain races --- it is not included in the first two models. It should be noted that each of the treated voters lived in uncontested congressional districts.

In model 4, we allow for the possibility that the treatment effect was different where the hurricane had greater intensity. In this model, \emph{Treated × 2018 × Relative Rainfall} allows the treatment effect to vary based on our proxy for hurricane strength. Finally, in model 5, we ask whether the treatment effect is different in counties where fewer polling places occurred (\emph{Treated × 2018 × Share of Expected Polling Places Open}). Model 5 includes controls for hurricane strength to tease apart the effect of polling place closures from hurricane strength. In models 4 and 5, control voters are assigned the value of their treated voter. While the regressions include the full set of uninteracted and interaction terms, we display only these variables' impact on the treatment estimate in table. In each model, robust standard errors are clustered at the level of the match (\protect\hyperlink{ref-Abadie2020}{Abadie and Spiess 2020}).

\begin{singlespace}
\input{"../temp/dind_full.tex"}
\end{singlespace}

The coefficient on \emph{Treated × 2018} in Table \ref{tab:full-dind} indicates that Hurricane Michael had a substantial depressive effect in 2018 among the treated voters. Models 1 -- 3 indicate that the hurricane reduced turnout in the treated counties by roughly 6.6 percentage points. Multiplied across the nearly 200 thousand registered voters in the treated counties indicates that some 13 thousand ballots went uncast due to the hurricane, a major effect in a year when a statewide senate race was decided by 10,033 votes.

Model 4 demonstrates that the turnout effect was not moderated by the strength of the hurricane. It should be noted, however, that there is not a tremendous amount of variation in relative rainfall among treated voters: the interquartile range for rainfall relative to the historical average stretches from 174\% to 200\%. Model 5 makes clear that the treatment effect was highly moderated by the share of polling places each county had to close. The estimated treatment effect ranges from -9.4 percentage points in Bay County (where 6 of 44 polling places were open, and the rainfall was 184\% of normal) to a \emph{positive} treatment of 4.7 percentage points in Franklin County, where 8 polling places were open compared to just 7 planned ones (and rainfall was just 120\% of normal). As we demonstrate in the Supplemental Information, a regression run only on Franklin County voters and their controls does indicate a positive treatment effect, implying that the Executive Order may have increased turnout where polling place closures were avoided.

\hypertarget{identifying-administrative-effects}{%
\subsection*{Identifying Administrative Effects}\label{identifying-administrative-effects}}
\addcontentsline{toc}{subsection}{Identifying Administrative Effects}

As discussed above, our primary strategy for isolating the administrative effects of the hurricane on turnout involves leveraging random assignment around county borders in the Florida panhandle in a double-matched triple-differences specification. Each voter inside the buffer in a treated county is matched with one voter in the buffer in an untreated county, once again using a genetic matching algorithm (\protect\hyperlink{ref-Sekhon2011}{Sekhon 2011}). These matches serve as our primary control voters. Ties are broken randomly, and matching is done with replacement.

In some cases, voters on either side of the border are in different congressional districts. This would pose a problem if these races were contested thanks to the potentially mobilizing effects of House races, but the entire buffer falls in uncontested congressional districts. This means that treated and untreated voters are not facing differential mobilization from congressional races. In constructing our set of primary control voters, equalizing individual-level exposure to Hurricane Michael is of paramount importance. As such, in this first match, we include only historical vote mode; voters' relative rainfall; and latitude and longitude. This ensures that treated and primary control voters will have similar past turnout trends and live near one another.

After matching, treated voters live an average of about 3.6 miles from their primary control voter. Importantly, the relative rainfall faced by treated and primary control voters is virtually identical: while rainfall during the period was 164\% of average for the primary control voters, it was 167\% of normal for the treated voters. We consider these differences sufficiently small to assume that, on average, treated and control voters were faced with identical individual-level effects.

Once our set of treated and primary control voters\footnote{For ease of notation, the combined set of treated and primary control voters will henceforth be referred to as ``Panhandle voters,'' while ``treated'' voters will distinguish Panhandle voters in treated counties from Panhandle voters in other counties. The use of ``Panhandle'' is a slight misnomer: it excludes Escambia, Santa Rosa, and Okaloosa Counties which are certainly part of the Florida Panhandle, as well as Jefferson County and others to its east which are sometimes considered part of the panhandle.} has been identified, each of these voters is matched with five other voters that lived in neither the treated nor the immediately surrounding counties. This matching procedure follows the same steps detailed in the Overall Turnout Effects section of this paper. Table \ref{tab:balance-secondary} presents the results of the secondary match. We improve along all characteristics.

\begin{singlespace}
\begin{table}[!h]

\caption{\label{tab:balance-tab-ll2}\label{tab:balance-secondary} Balance Table for Secondary Match}
\centering
\resizebox{\linewidth}{!}{
\begin{tabular}[t]{lllllrrrr}
\toprule
\multicolumn{1}{c}{ } & \multicolumn{2}{c}{Means: Unmatched Data} & \multicolumn{2}{c}{Means: Matched Data} & \multicolumn{4}{c}{Percent Improvement} \\
\cmidrule(l{3pt}r{3pt}){2-3} \cmidrule(l{3pt}r{3pt}){4-5} \cmidrule(l{3pt}r{3pt}){6-9}
 & Treated & Control & Treated & Control & Mean Diff & eQQ Med & eQQ Mean & eQQ Max\\
\midrule
\%White & 71.7\% & 62.3\% & 71.7\% & 71.7\% & 100.00 & 100.00 & 100.00 & 100.00\\
\% Black & 23.3\% & 13.1\% & 23.3\% & 23.3\% & 100.00 & 100.00 & 100.00 & 100.00\\
\% Latino & 1.4\% & 17.4\% & 1.4\% & 1.4\% & 100.00 & 100.00 & 100.00 & 100.00\\
\% Asian & 0.5\% & 2.0\% & 0.5\% & 0.5\% & 100.00 & 100.00 & 100.00 & 100.00\\
\% Female & 52.7\% & 52.4\% & 52.7\% & 52.7\% & 100.00 & 100.00 & 100.00 & 100.00\\
\% Male & 45.6\% & 44.9\% & 45.6\% & 45.6\% & 100.00 & 100.00 & 100.00 & 100.00\\
Age & 52.9 & 52.5 & 52.9 & 52.9 & 98.12 & 82.32 & 87.10 & 87.22\\
\% Democrat & 46.4\% & 37.1\% & 46.4\% & 46.4\% & 100.00 & 100.00 & 100.00 & 100.00\\
\% Republican & 38.7\% & 35.0\% & 38.7\% & 38.7\% & 100.00 & 100.00 & 100.00 & 100.00\\
\% with Some College & 62.9\% & 75.1\% & 62.9\% & 62.9\% & 99.98 & 99.30 & 97.16 & 82.78\\
Median Income & \$45,913 & \$62,941 & \$45,913 & \$45,928 & 99.91 & 99.03 & 96.22 & 80.63\\
\bottomrule
\end{tabular}}
\end{table}
\end{singlespace}

In Figure \ref{fig:trip-diff-plot} we present the plotted turnout trends from the treatment, primary control, and secondary control groups returned by the matching exercise. Figure \ref{fig:trip-diff-plot} makes clear that the turnout gap between treated and primary control voters was largely constant in the base period, although treated voters' turnout was higher than their controls' in 2016. Insofar as the ``natural'' turnout of treated voters was increasing relative to that of their primary controls in 2016 and 2018, our model will be biased against finding a negative treatment effect, making any negative treatment effect conservative. The turnout gap between Panhandle and secondary control voters is constant across the base period.

\begin{figure}[h]

{\centering \includegraphics{hurricane_michael_files/figure-latex/tripd-to-chunk-1} 

}

\caption{\label{fig:trip-diff-plot}General Election Turnout for Treated, Primary Control, and Secondary Control Voters, 2010 -- 2018}\label{fig:tripd-to-chunk}
\end{figure}

Disentangling the administrative and individual effects of the storm requires the estimation of the triple-differences model. This model is estimated by Equation (1).

\begin{gather}
\label{eq:1}
v_{it}=\beta_0+\beta_1Weather Treatment_{i}+\beta_22018_{t}+\beta_3Weather Treatment_{i}\times 2018_{t} + \nonumber \\
\beta_4Administrative Treatment_{i} + \beta_5Administrative Treatment_{i}\times 2018_{t} + \\
\delta{Y}_{it} + \delta{Z}_{i} + \mathcal{E}_{it}. \nonumber
\end{gather}

Individual \emph{i}'s turnout (\emph{v}) in year \emph{t} is a function of the year and their location. In the equation, \emph{\(\beta\)\textsubscript{1}Panhandle\textsubscript{i}} measures the historical difference between voters in the panhandle and the rest of the state. \emph{\(\beta\)\textsubscript{2}2018\textsubscript{t}} measures the statewide change in turnout in 2018 from the baseline, while \emph{\(\beta\)\textsubscript{3}Panhandle\textsubscript{i} × 2018\textsubscript{t}} tests whether turnout changed differently in 2018 in the panhandle than it did elsewhere. \emph{\(\beta\)\textsubscript{3}Panhandle\textsubscript{i} × 2018\textsubscript{t}}, therefore, is our estimation of the individual-level, or weather related, effect of the hurricane. \emph{\(\beta\)\textsubscript{4}Treated\textsubscript{i}} measures the historical difference between treated and primary control voters, and \emph{\(\beta\)\textsubscript{5}Treated\textsubscript{i} × 2018\textsubscript{t}} tests whether the causal effect of the storm was different for treated voters than for their primary controls. This, then, is the estimated administrative effect of living in a county covered by the Executive Order. The matrix \emph{\(\delta\)Y\textsubscript{i}} includes the measures for relative rainfall and polling place closures interacted with treatment, panhandle, and 2018 dummies. The matrix \emph{\(\delta\)Z\textsubscript{i}} includes the covariates used in the matching procedure.

Table \ref{tab:trip-diff} presents the results of this model, again fit using an ordinary least squares specification. Model 1 does not include \emph{\(\delta\)Z\textsubscript{i}}, while the matrix is included in Models 2 and 3. Model 3 also includes estimates for congressional district competitiveness in 2018. Finally, in Model 4, we once again investigate whether the treatment effect was moderated by polling place closures and relative rainfall. While the models include the full matrix \emph{\(\delta\)Y\textsubscript{i}}, we display only rain and polling place closures' influence on the treatment effect in the table for the sake of legibility. Robust standard errors are clustered at the level of the original treated voter from which the primary and secondary controls arise.

\begin{singlespace}

\input{"../temp/triple_diff.tex"}
\end{singlespace}

The coefficients on \emph{Panhandle × 2018} and \emph{Treated × 2018} are of most substantive interest here. The coefficient on \emph{Panhandle × 2018} indicates that turnout for the primary control voters in 2018 was not \sout{statistically} significantly different than the 2018 turnout of the secondary controls, Hurricane Michael notwithstanding. Given that these counties were not covered by the Executive Order because they were not in the direct path of the storm, this lack of a turnout effect is unsurprising.

There was, however, a negative treatment effect for voters just inside the treated counties. \emph{Treated × 2018} in models 1--3 indicates that, for voters just inside the treated counties, turnout was depressed relative to their primary controls by 1.9 percentage points. This 1.9 percentage point decrease in turnout for voters inside the treated counties is the administrative effect on turnout.

Model 4 once again demonstrates that these effects were moderated by polling place consolidation and the strength of the storm---with polling place consolidation having a far larger impact. In this set of treated voters, there is a negative relationship between polling place consolidation and relative rainfall. Treated voters in Bay County (where 6 of 44 polling places were open) saw rainfall 155\% of normal; in Gadsden and Liberty Counties where the expected number of polling places were open, by contrast, treated voters saw rainfall that was 213\% and 229\% of normal, respectively. Multiplying out the coefficients from model 4 in Table \ref{tab:trip-diff} results in an estimated administrative treatment effects ranging from -6.4 points in Bay County to +0.35 points in Gadsden. Once again, we see that county-level polling place consolidation had a far larger influence on turnout than the storm itself.

Importantly, the decomposed administrative- and individual- effects estimated in Table \ref{tab:trip-diff} are the average treatment effect on the treated voters (ATT). Nevertheless, the administrative effect of -1.9 percentage points is substantively quite large. Despite the efforts of Executive Order 18-283, the administrative costs imposed by Hurricane Michael meaningfully depressed turnout. As model 4 indicates, however, the Executive Order may have \emph{increased} turnout where counties were able to keep the bulk of their polling places open. \textbf{Does this paragraph lead us to revising the summary of the paper in the abstract?}

\hypertarget{shifting-vote-modes}{%
\section*{Shifting Vote Modes}\label{shifting-vote-modes}}
\addcontentsline{toc}{section}{Shifting Vote Modes}

Having established that turnout was substantially depressed in the treated counties and that a non-trivial amount of the depression arose from administrative costs, we turn to a new question: did the storm shift \emph{how} people cast their ballots? We know that Executive Order 18-283 loosened restrictions on early and mail balloting; we therefore expect that, relative to the rest of the state, a higher share of ballots in the treated counties cast their ballots in one of these ways.

We return to the matches produced earlier in this paper, where every voter in the treated counties was matched with five voters elsewhere in the state. Figure \ref{fig:vote-mode} demonstrates the share of registered voters that cast a ballot either at the polling place, early in person, or absentee in each general election from the past decade. In each case, the denominator is the number of registered voters in 2018.

\begin{figure}[h]

{\centering \includegraphics{hurricane_michael_files/figure-latex/vote-mode-chunk-1} 

}

\caption{\label{fig:vote-mode}Marginal Effect of Relocated Polling Place on Vote Mode}\label{fig:vote-mode-chunk}
\end{figure}

Figure \ref{fig:vote-mode} makes clear that the decline in turnout was a product of lower turnout on election day and via absentee voting. It seems, however, that early voting was actually higher in the treated counties due to Hurricane Michael.

To more directly estimate the effect of Hurricane Michael and the Executive Order on vote-mode, we measure how far each treated voter lived from the closest planned polling place and the polling place that actually opened on election day. Using a multinomial logistic regression, we test whether increasing the difference between this distance is related to vote-mode or abstention in 2018. In addition to the difference between expected and actual distance to the closest polling place, we include other covariates. We measure how far a voter lived from her closest \emph{planned} polling place, in case voters in more remote parts of the counties generally voted differently in 2018 than other voters. We include other covariates for individual characteristics such as race, age, and partisan affiliation. We also include dummies indicating how (or whether) each voter participated in the 2012 -- 2016 general elections.

Because the coefficients from the mulinomial logistic regression are difficult to interpret on their own, we include here the marginal effects plots from this model (the full regression table can be found in the Supplementary Information). Figure \ref{fig:marg-multi} presents the marginal effect of the change in distance to the nearest polling place on vote method while keeping all other covariates in the model at their means.

\begin{figure}[h]

{\centering \includegraphics{hurricane_michael_files/figure-latex/marg-multi-1} 

}

\caption{\label{fig:marg-multi}Marginal Effect of Changed Distance to Polling Place on 2018 Vote Mode}\label{fig:marg-multi}
\end{figure}

Figure \ref{fig:marg-multi} indicates that, as voters suddenly had to travel further to the nearest polling place, they were substantially less likely to vote in person on election day (``In Person (ED)''). The bulk of these voters \emph{did not} shift to absentee voting or early in-person voting; rather, they were much more likely to abstain from casting a ballot at all. Thus, although administrators took steps to make early and mail voting easier, these efforts were not particularly effective.

\hypertarget{discussion-and-conclusion}{%
\section*{Discussion and Conclusion}\label{discussion-and-conclusion}}
\addcontentsline{toc}{section}{Discussion and Conclusion}

Election Day in the United States consistently falls near the end of hurricane season. Hurricane Michael made landfall on October 10, 2018, less than a month before the highest-turnout midterm election in a century. Hurricane Sandy struck New York and New Jersey just days before the midterm elections in 2012, wreaking immense havoc. Hurricane Matthew struck the Southeastern United States weeks before the 2016 presidential election, killing dozens and causing more than \$2.5 billion in damages. \protect\hyperlink{ref-Mann2006}{Mann and Emanuel} (\protect\hyperlink{ref-Mann2006}{2006}) and others have linked Atlantic hurricanes to climate change, indicating that these disruptions to election day activity are likely to increase in coming years. Understanding how storms of this nature impact turnout --- and whether states' responses are sufficient to recoup turnout --- is therefore vitally important, particularly in swing states such as Florida and North Carolina that are subject to severe coastal natural disasters.

As this paper demonstrates, Florida's response to Hurricane Michael was not particularly effective: although Governor Scott increased access to early and mail voting in eight counties, mail balloting use in these areas actually \emph{dropped} relative to the rest of the state (see Figure \ref{fig:vote-mode}). Despite the Executive Order, turnout dropped substantially for voters who suddenly were faced with long distances to the closest polling places. These voters did not move to vote-by-mail options in appreciable numbers.

This is disheartening. Not only did the Executive Order fail to combat the negative individual-level effects of the hurricane on turnout, it was also insufficient at mitigating the negative administrative effects of closed polling places. Clearly, loosening restrictions on where mail ballots could be sent and how they could be returned was not enough. Without the Executive Order, polling places would still have been moved because some had been destroyed, but the loosened restrictions on other modes would not have been accessible. Thus, the Executive Order likely reduced the administrative costs of voting. Nevertheless, these administrative effects remained quite large and were responsible for nearly half the depressive effect of the storm for voters living at the outer edges of the covered counties.

The data at hand cannot explain why the Executive Order was ineffective at neutralizing the administrative effects of the hurricane. The timing of the Executive Order, however, might shed some light. Although the hurricane made landfall on October 10, the Executive Order was not signed until more than a week later, on October 18 --- fewer than three weeks before the November 6 general election. This left little time for an effective public education campaign, perhaps limiting the number of voters who learned and took advantage of the changed rules. We found very few news articles detailing the changes and making the information easily available to voters (but see \protect\hyperlink{ref-WJHG2018}{\emph{WJHG - Panama City} 2018}; \protect\hyperlink{ref-Vasquez2018}{Vasquez 2018}; \protect\hyperlink{ref-McDonald2018}{McDonald 2018}; \protect\hyperlink{ref-Fineout2018}{Fineout 2018}), and what information did get published often listed only relocated polling places with no information about loosened mail voting restrictions (see, for instance, \protect\hyperlink{ref-gadsdentimes2018}{\emph{Gadsden Times} 2018}). It is possible, of course, that local televised news communicated the changes to viewers; however, based on our search of published information, that information would have been difficult to find for voters who missed the televised news. We found no evidence that the Florida Times-Union (the largest paper in Northern Florida) or the Tampa Bay Times (the largest paper in the state) published any articles detailing the changes brought about by the Executive Order.

Future research will no doubt leverage pre-existing administrative regimes to understand the sorts of voting environments least susceptible to disruption, like those following from the coronavirus in the context of the 2020 elections --- but such research will necessarily be backward looking. The experience of Hurricane Michael, on the other hand, gives us important insight about how an emergency that closes polling places will structure turnout. Our research on Executive Order 18-283 makes clear that loosened restrictions on mail voting alone cannot combat the negative turnout effects of shuttered polling places.

\sout{The novel coronavirus will perhaps lower turnout even if election administrators respond perfectly. Voting might be low on a list of priorities for individuals who are caring for ailing loved ones, grieving, or dealing with economic crises. Nevertheless, COVID-19 will also pose administrative hurdles to voting: consolidated or relocated polling places, reliance on a vote-by-mail system unfamiliar to many voters, or longer wait times as the number of voters allowed into a polling place at once might all reduce turnout. As administrators consider easing vote-by-mail restrictions, they must look to the case of Florida in 2018. More must be done than simply change the rules; otherwise, the administrative effects of COVID-19 will magnify the individual effects of this public health crisis on voter turnout.}

\newpage

\hypertarget{references}{%
\section*{References}\label{references}}
\addcontentsline{toc}{section}{References}

\hypertarget{refs}{}
\begin{CSLReferences}{1}{0}
\leavevmode\hypertarget{ref-Abadie2020}{}%
Abadie, Alberto, and Jann Spiess. 2020. {``Robust {Post}-{Matching Inference}.''} \emph{Journal of the American Statistical Association} 0 (0): 1--13. \url{https://doi.org/10.1080/01621459.2020.1840383}.

\leavevmode\hypertarget{ref-Brady2011}{}%
Brady, Henry, and John McNulty. 2011. {``Turning Out to Vote: {The Costs} of {Finding} and {Getting} to the {Polling Place}.''} \emph{American Political Science Review} 105 (1): 115--34.

\leavevmode\hypertarget{ref-Burden2014}{}%
Burden, Barry C., David T. Canon, Kenneth R. Mayer, and Donald P. Moynihan. 2014. {``Election {Laws}, {Mobilization}, and {Turnout}: {The Unanticipated Consequences} of {Election Reform}.''} \emph{American Journal of Political Science} 58 (1): 95--109. \url{https://doi.org/10.1111/ajps.12063}.

\leavevmode\hypertarget{ref-Cantoni2020}{}%
Cantoni, Enrico. 2020. {``A {Precinct Too Far}: {Turnout} and {Voting Costs}.''} \emph{American Economic Journal: Applied Economics} 12 (1): 61--85.

\leavevmode\hypertarget{ref-Chamberlain2021}{}%
Chamberlain, Scott. 2021. \emph{Rnoaa: '{NOAA}' {Weather Data} from {R}}. \url{https://CRAN.R-project.org/package=rnoaa}.

\leavevmode\hypertarget{ref-Cooperman2017}{}%
Cooperman, Alicia. 2017. {``Randomization {Inference} with {Rainfall Data}: {Using Historical Weather Patterns} for {Variance Estimation}.''} \emph{Political Analysis} 25 (3): 277--88.

\leavevmode\hypertarget{ref-Dyck2005}{}%
Dyck, Joshua, and James Gimpel. 2005. {``Distance, {Turnout}, and the {Convenience} of {Voting}.''} \emph{Social Science Quarterly} 86 (3): 531--48.

\leavevmode\hypertarget{ref-Fineout2018}{}%
Fineout, Gary. 2018. {``Florida to Bend Voting Rules in Counties Hit by Hurricane.''} \emph{Northwest Florida Daily News}, October 18, 2018. \url{https://www.nwfdailynews.com/news/20181018/florida-to-bend-voting-rules-in-counties-hit-by-hurricane}.

\leavevmode\hypertarget{ref-Fraga2010}{}%
Fraga, Bernard, and Eitan Hersh. 2010. {``Voting {Costs} and {Voter Turnout} in {Competitive Elections}.''} \emph{Quarterly Journal of Political Science} 5: 339--56. https://doi.org/\url{http://dx.doi.org/10.1561/100.00010093_supp}.

\leavevmode\hypertarget{ref-gadsdentimes2018}{}%
\emph{Gadsden Times}. 2018. {``Changes in Polling Places at Three Locations,''} October 30, 2018. \url{https://www.gadsdentimes.com/news/20181030/changes-in-polling-places-at-three-locations}.

\leavevmode\hypertarget{ref-Garcia-Rodriguez2020}{}%
Garcia-Rodriguez, Abian, and Paul Redmond. 2020. {``Rainfall, Population Density and Voter Turnout.''} \emph{Electoral Studies} 64 (April): 102128. \url{https://doi.org/10.1016/j.electstud.2020.102128}.

\leavevmode\hypertarget{ref-Gatrell2002}{}%
Gatrell, Jay, and Gregory Bierly. 2002. {``Weather and {Voter Turnout}: {Kentucky Primary} and {General Elections}, 1990-2000.''} \emph{Southeastern Geographer} 42 (1): 114--34.

\leavevmode\hypertarget{ref-Hansford2010}{}%
Hansford, Thomas, and Brad Gomez. 2010. {``Estimating the {Electoral Effects} of {Voter Turnout}.''} \emph{American Political Science Review} 104: 268--88.

\leavevmode\hypertarget{ref-Haspel2005}{}%
Haspel, Moshe, and H. Gibbs Knotts. 2005. {``Location, {Location}, {Location}: {Precinct Placement} and the {Costs} of {Voting}.''} \emph{Journal of Politics} 67 (2): 560--73.

\leavevmode\hypertarget{ref-Kaplan2020}{}%
Kaplan, Ethan, and Haishan Yuan. 2020. {``Early {Voting Laws}, {Voter Turnout}, and {Partisan Vote Composition}: {Evidence} from {Ohio}.''} \emph{American Economic Journal: Applied Economics} 12 (1): 32--60.

\leavevmode\hypertarget{ref-Keele2015}{}%
Keele, Luke J., and Rocío Titiunik. 2015/ed. {``Geographic {Boundaries} as {Regression Discontinuities}.''} \emph{Political Analysis} 23 (1): 127--55. \url{https://doi.org/10.1093/pan/mpu014}.

\leavevmode\hypertarget{ref-Keele2015a}{}%
Keele, Luke, Rocío Titiunik, and José R. Zubizarreta. 2015. {``Enhancing a Geographic Regression Discontinuity Design Through Matching to Estimate the Effect of Ballot Initiatives on Voter Turnout.''} \emph{Journal of the Royal Statistical Society: Series A (Statistics in Society)} 178 (1): 223--39. \url{https://doi.org/10.1111/rssa.12056}.

\leavevmode\hypertarget{ref-Kitamura2020}{}%
Kitamura, Shuhei, and Tetsuya Matsubayashi. 2020. {``Dynamic {Voting}.''}

\leavevmode\hypertarget{ref-Kropf2012}{}%
Kropf, Martha, and David Kimball. 2012. \emph{Helping {America Vote}: {The Limits} of {Election Reform}}. {New York}: {Routledge}.

\leavevmode\hypertarget{ref-Larocca2011}{}%
Larocca, Roger, and John S. Klemanski. 2011. {``U.{S}. {State Election Reform} and {Turnout} in {Presidential Elections}.''} \emph{State Politics \& Policy Quarterly} 11 (1): 76--101. \url{https://doi.org/10.1177/1532440010387401}.

\leavevmode\hypertarget{ref-Mann2006}{}%
Mann, Michael E., and Kerry A. Emanuel. 2006. {``Atlantic Hurricane Trends Linked to Climate Change.''} \emph{Eos, Transactions American Geophysical Union} 87 (24): 233--41. \url{https://doi.org/10.1029/2006EO240001}.

\leavevmode\hypertarget{ref-McDonald2018}{}%
McDonald, Zack. 2018. {``Bay Voters Getting 5 'Mega Voting' Sites.''} \emph{Panama City News Herald}, October 23, 2018. \url{https://www.newsherald.com/news/20181023/bay-voters-getting-5-mega-voting-sites}.

\leavevmode\hypertarget{ref-McNulty2009}{}%
McNulty, John, Conor Dowling, and Margaret Ariotti. 2009. {``Driving {Saints} to {Sin}: {How Increasing} the {Difficulty} of {Voting Dissuades Even} the {Most Motivated Voters}.''} \emph{Political Analysis} 17 (4): 435--55.

\leavevmode\hypertarget{ref-Morris2021}{}%
Morris, Kevin, and Peter Miller. 2021. {``Voting in a {Pandemic}: {COVID}-19 and {Primary Turnout} in {Milwaukee}, {Wisconsin}.''} \emph{Urban Affairs Review}, April, 10780874211005016. \url{https://doi.org/10.1177/10780874211005016}.

\leavevmode\hypertarget{ref-Nyhan2017}{}%
Nyhan, Brendan, Christopher Skovron, and Rocío Titiunik. 2017. {``Differential {Registration Bias} in {Voter File Data}: {A Sensitivity Analysis Approach}.''} \emph{American Journal of Political Science} 61 (3): 744--60. \url{https://doi.org/10.1111/ajps.12288}.

\leavevmode\hypertarget{ref-Parks2018}{}%
Parks, Miles. 2018. {``After {Hurricane Michael}, {Voting} '{Is The Last Thing On Their Minds}'.''} \emph{NPR.org}, October 25, 2018. \url{https://www.npr.org/2018/10/25/659819848/after-hurricane-michael-voting-is-the-last-thing-on-their-minds}.

\leavevmode\hypertarget{ref-Persson2014}{}%
Persson, Mikael, Anders Sundell, and Richard Öhrvall. 2014. {``Does {Election Day Weather Affect Voter Turnout}? {Evidence} from {Swedish Elections}.''} \emph{Electoral Studies} 33: 335--42.

\leavevmode\hypertarget{ref-Rallings2003}{}%
Rallings, Colin, Michael Thrasher, and Roman Borisyuk. 2003. {``Seasonal {Factors}, Voter Fatigue, and the Costs of Voting.''} \emph{Electoral Studies} 22: 65--79.

\leavevmode\hypertarget{ref-Ricardson1996}{}%
Ricardson, Lilliard, and Grant Neeley. 1996. {``The {Impact} of {Early Voting} on {Turnout}: {The} 1994 {Elections} in {Tennessee}.''} \emph{State \& Local Government Review} 28 (3): 173--79.

\leavevmode\hypertarget{ref-Sekhon2009}{}%
Sekhon, Jasjeet. 2009. {``Opiates for the {Matches}: {Matching Methods} for {Causal Inference}.''} \emph{Annual Review of Political Science} 12: 487--508.

\leavevmode\hypertarget{ref-Sekhon2011}{}%
---------. 2011. {``Multivariate and {Propensity Score Matching Software} with {Automated Balance Optimization}: {The Matching} Package for {R}.''} \emph{Journal of Statistical Software} 42 (1): 1--52. \url{https://doi.org/10.18637/jss.v042.i07}.

\leavevmode\hypertarget{ref-Stein2015}{}%
Stein, Robert. 2015. {``Election {Administration During National Disasters} and {Emergencies}: {Hurricane Sandy} and the 2012 {Election}.''} \emph{Election Law Journal} 14: 66--73.

\leavevmode\hypertarget{ref-Vasquez2018}{}%
Vasquez, Savannah. 2018. {``{HURRICANE MICHAEL}: {How} to Vote in {Gulf County}.''} \emph{The Star}, October 18, 2018. \url{https://www.starfl.com/news/20181018/hurricane-michael-how-to-vote-in-gulf-county}.

\leavevmode\hypertarget{ref-Walker2019}{}%
Walker, Hannah L., Michael C. Herron, and Daniel A. Smith. 2019. {``Early {Voting Changes} and {Voter Turnout}: {North Carolina} in the 2016 {General Election}.''} \emph{Political Behavior} 41 (4): 841--69. \url{https://doi.org/10.1007/s11109-018-9473-5}.

\leavevmode\hypertarget{ref-WJHG2018}{}%
\emph{WJHG - Panama City}. 2018. {``Bay {County Hurricane Michael Recovery Information},''} October 31, 2018. \url{https://www.wjhg.com/content/news/Bay-County–498037961.html}.

\end{CSLReferences}

\end{document}
